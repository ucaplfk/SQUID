\section{\label{sec:level1}Discussion and Conclusion}
Sinusoidal oscillations of the $I_C$ as a function of the $B$ strength was not observed for this fabricated device. Conversely, there are a number of factors to consider and effects which may be restricting the characteristic SQUID behaviour. Firstly, the JJs dimensions of $l$ and $w$ are greater than the $\xi$ for $Nb$. Although the dimensions do not exceed 3$\xi$, higher harmonic $\sin (N\delta )$ components may be present $^[$\citep{Clarke2005TheHandbook}$^]$. Furthermore, Fig. (\ref{fig:IVplot}) of $I-V$ for the SQUID is hysteric. The production of "hot spots" on the device due to heating effects when  $I\approx$ $I_{SW}$ produces a incoherent CPR $^[$\citep{Podd2007Micro-SQUIDsJunctions}$^]$. Despite observing a large peak at $\approx$ 0.5 (mT)\textsuperscript{-1} for the FFT signal of positive $I_C$ and the linear fit of $R$, these results are not sufficient to verify the sinusoidal relationship between $I_C$ and $\delta$ for an increasing $B$ field. This is due to the absence of a large amplitude peak at $\approx$ 0.5 (mT)\textsuperscript{-1} when the FFT signals are plotted for the $B$ field decreasing from 5 mT to 0. 

In future fabrication and measurement of nanoSQUIDs the dimensions of the bridge should be reduced to as close to $\xi$ as possible without breaking the weak-link JJ. The size of the SQUID loop could be reduced as decreasing the area of the loop increases the device's sensitivity to changes in the magnetic flux. When measuring the $I_C$ oscillations as a function of the applied $B$ field the maximum value of $I-V$ slope should not be hysteric. The hysteric behaviour is expected to be absent for devices where the $I_C$< 25 $\mu$A. Yet for this device $I_C=(5.31 \pm 0.05)$ $\mu$A is measured when $B\approx$0 and the hysteresis is present $^[$\citep{Hao2015FabricationJunctions}$^]$. This may be a result of the low $\approx$ 300 mK temperature of the device so varying the temperature is a possible way to remove the hystersis. Further measurements should be completed over a shortened range of $B$ where more detailed measurements of $I_C$ are required to determine with confidence the observation of magnetic field dependent $I_C$ oscillations.  




