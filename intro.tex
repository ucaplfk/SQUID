\section{\label{sec:level1}Introduction}
In recent years the miniaturisation, to micro and nano-chip scales, has been the focus for development of various quantum sensing devices. SQUIDs are devices which enable high precision measurement of small changes in the magnetic flux due to period oscillations of the critical current, $I_C$ in an applied magnetic flux, $\Phi$ $^[$\citep{Podd2007Micro-SQUIDsJunctions}${}^]$. Developing devices which can measure the magnetisation of particles of the order of nanometers has applications in a range of fields including biomedicine, telecommunication and quantum computing $^[$\citep{Hao2015FabricationJunctions}${}^]$.

Typical fabrication techniques produce a dc nanoSQUID consisting of two parallel Dayem Josephson junctions (JJs) connected to a superconducting loop $^[$\citep{Hao2015FabricationJunctions}${}^]$. These quantum sensing devices exploit the Josephson effect and flux quantisation. At temperatures below $T_C$ for superconducting, Josephson tunneling of Cooper pairs through the junction between two superconductors creates a supper current $I_S=I_0\sin\delta$. The phase difference between the superconducting materials is $\delta$. For Type 1 superconductors below $T_C$ the magnetic flux in the loop is expelled. However, in the case of a Type 2 superconductor such as $Nb$ vortices of magnetic flux can penetrate through in quantised units of the flux quantum ${\Phi _0}$=h/2e $^[$\citep{Clarke2005TheHandbook}${}^]$.